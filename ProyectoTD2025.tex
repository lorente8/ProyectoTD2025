%  LaTeX support: latex@mdpi.com
%  For support, please attach all files needed for compiling as well as the log file, and specify your operating system, LaTeX version, and LaTeX editor.

%=================================================================
% pandoc conditionals added to preserve backwards compatibility with previous versions of rticles

\documentclass[,,,oneauthor,pdftex]{Definitions/mdpi}


%% Some pieces required from the pandoc template
\setlist[itemize]{leftmargin=*,labelsep=5.8mm}
\setlist[enumerate]{leftmargin=*,labelsep=4.9mm}


%--------------------
% Class Options:
%--------------------

%---------
% article
%---------
% The default type of manuscript is "article", but can be replaced by:
% abstract, addendum, article, book, bookreview, briefreport, casereport, comment, commentary, communication, conferenceproceedings, correction, conferencereport, entry, expressionofconcern, extendedabstract, datadescriptor, editorial, essay, erratum, hypothesis, interestingimage, obituary, opinion, projectreport, reply, retraction, review, perspective, protocol, shortnote, studyprotocol, systematicreview, supfile, technicalnote, viewpoint, guidelines, registeredreport, tutorial
% supfile = supplementary materials

%----------
% submit
%----------
% The class option "submit" will be changed to "accept" by the Editorial Office when the paper is accepted. This will only make changes to the frontpage (e.g., the logo of the journal will get visible), the headings, and the copyright information. Also, line numbering will be removed. Journal info and pagination for accepted papers will also be assigned by the Editorial Office.

%------------------
% moreauthors
%------------------
% If there is only one author the class option oneauthor should be used. Otherwise use the class option moreauthors.

%---------
% pdftex
%---------
% The option pdftex is for use with pdfLaTeX. Remove "pdftex" for (1) compiling with LaTeX & dvi2pdf (if eps figures are used) or for (2) compiling with XeLaTeX.

%=================================================================
% MDPI internal commands - do not modify
\firstpage{1}
\makeatletter
\setcounter{page}{\@firstpage}
\makeatother
\pubvolume{1}
\issuenum{1}
\articlenumber{0}
\pubyear{2023}
\copyrightyear{2023}
%\externaleditor{Academic Editor: Firstname Lastname}
\datereceived{ }
\daterevised{ } % Comment out if no revised date
\dateaccepted{ }
\datepublished{ }
%\datecorrected{} % For corrected papers: "Corrected: XXX" date in the original paper.
%\dateretracted{} % For corrected papers: "Retracted: XXX" date in the original paper.
\hreflink{https://doi.org/} % If needed use \linebreak
%\doinum{}
%\pdfoutput=1 % Uncommented for upload to arXiv.org

%=================================================================
% Add packages and commands here. The following packages are loaded in our class file: fontenc, inputenc, calc, indentfirst, fancyhdr, graphicx, epstopdf, lastpage, ifthen, float, amsmath, amssymb, lineno, setspace, enumitem, mathpazo, booktabs, titlesec, etoolbox, tabto, xcolor, colortbl, soul, multirow, microtype, tikz, totcount, changepage, attrib, upgreek, array, tabularx, pbox, ragged2e, tocloft, marginnote, marginfix, enotez, amsthm, natbib, hyperref, cleveref, scrextend, url, geometry, newfloat, caption, draftwatermark, seqsplit
% cleveref: load \crefname definitions after \begin{document}

%=================================================================
% Please use the following mathematics environments: Theorem, Lemma, Corollary, Proposition, Characterization, Property, Problem, Example, ExamplesandDefinitions, Hypothesis, Remark, Definition, Notation, Assumption
%% For proofs, please use the proof environment (the amsthm package is loaded by the MDPI class).

%=================================================================
% Full title of the paper (Capitalized)
\Title{ProyectoTD2025}

% MDPI internal command: Title for citation in the left column
\TitleCitation{ProyectoTD2025}

% Author Orchid ID: enter ID or remove command
%\newcommand{\orcidauthorA}{0000-0000-0000-000X} % Add \orcidA{} behind the author's name
%\newcommand{\orcidauthorB}{0000-0000-0000-000X} % Add \orcidB{} behind the author's name


% Authors, for the paper (add full first names)
\Author{$^{}$}


%\longauthorlist{yes}


% MDPI internal command: Authors, for metadata in PDF
\AuthorNames{}

% MDPI internal command: Authors, for citation in the left column

% Affiliations / Addresses (Add [1] after \address if there is only one affiliation.)
\address{%
}

% Contact information of the corresponding author
\corres{Correspondence: }

% Current address and/or shared authorship








% The commands \thirdnote{} till \eighthnote{} are available for further notes

% Simple summary

%\conference{} % An extended version of a conference paper

% Abstract (Do not insert blank lines, i.e. \\)


% Keywords

% The fields PACS, MSC, and JEL may be left empty or commented out if not applicable
%\PACS{J0101}
%\MSC{}
%\JEL{}

%%%%%%%%%%%%%%%%%%%%%%%%%%%%%%%%%%%%%%%%%%
% Only for the journal Diversity
%\LSID{\url{http://}}

%%%%%%%%%%%%%%%%%%%%%%%%%%%%%%%%%%%%%%%%%%
% Only for the journal Applied Sciences

%%%%%%%%%%%%%%%%%%%%%%%%%%%%%%%%%%%%%%%%%%

%%%%%%%%%%%%%%%%%%%%%%%%%%%%%%%%%%%%%%%%%%
% Only for the journal Data



%%%%%%%%%%%%%%%%%%%%%%%%%%%%%%%%%%%%%%%%%%
% Only for the journal Toxins


%%%%%%%%%%%%%%%%%%%%%%%%%%%%%%%%%%%%%%%%%%
% Only for the journal Encyclopedia


%%%%%%%%%%%%%%%%%%%%%%%%%%%%%%%%%%%%%%%%%%
% Only for the journal Advances in Respiratory Medicine
%\addhighlights{yes}
%\renewcommand{\addhighlights}{%

%\noindent This is an obligatory section in “Advances in Respiratory Medicine”, whose goal is to increase the discoverability and readability of the article via search engines and other scholars. Highlights should not be a copy of the abstract, but a simple text allowing the reader to quickly and simplified find out what the article is about and what can be cited from it. Each of these parts should be devoted up to 2~bullet points.\vspace{3pt}\\
%\textbf{What are the main findings?}
% \begin{itemize}[labelsep=2.5mm,topsep=-3pt]
% \item First bullet.
% \item Second bullet.
% \end{itemize}\vspace{3pt}
%\textbf{What is the implication of the main finding?}
% \begin{itemize}[labelsep=2.5mm,topsep=-3pt]
% \item First bullet.
% \item Second bullet.
% \end{itemize}
%}


%%%%%%%%%%%%%%%%%%%%%%%%%%%%%%%%%%%%%%%%%%

% Pandoc syntax highlighting
\usepackage{color}
\usepackage{fancyvrb}
\newcommand{\VerbBar}{|}
\newcommand{\VERB}{\Verb[commandchars=\\\{\}]}
\DefineVerbatimEnvironment{Highlighting}{Verbatim}{commandchars=\\\{\}}
% Add ',fontsize=\small' for more characters per line
\usepackage{framed}
\definecolor{shadecolor}{RGB}{248,248,248}
\newenvironment{Shaded}{\begin{snugshade}}{\end{snugshade}}
\newcommand{\AlertTok}[1]{\textcolor[rgb]{0.94,0.16,0.16}{#1}}
\newcommand{\AnnotationTok}[1]{\textcolor[rgb]{0.56,0.35,0.01}{\textbf{\textit{#1}}}}
\newcommand{\AttributeTok}[1]{\textcolor[rgb]{0.13,0.29,0.53}{#1}}
\newcommand{\BaseNTok}[1]{\textcolor[rgb]{0.00,0.00,0.81}{#1}}
\newcommand{\BuiltInTok}[1]{#1}
\newcommand{\CharTok}[1]{\textcolor[rgb]{0.31,0.60,0.02}{#1}}
\newcommand{\CommentTok}[1]{\textcolor[rgb]{0.56,0.35,0.01}{\textit{#1}}}
\newcommand{\CommentVarTok}[1]{\textcolor[rgb]{0.56,0.35,0.01}{\textbf{\textit{#1}}}}
\newcommand{\ConstantTok}[1]{\textcolor[rgb]{0.56,0.35,0.01}{#1}}
\newcommand{\ControlFlowTok}[1]{\textcolor[rgb]{0.13,0.29,0.53}{\textbf{#1}}}
\newcommand{\DataTypeTok}[1]{\textcolor[rgb]{0.13,0.29,0.53}{#1}}
\newcommand{\DecValTok}[1]{\textcolor[rgb]{0.00,0.00,0.81}{#1}}
\newcommand{\DocumentationTok}[1]{\textcolor[rgb]{0.56,0.35,0.01}{\textbf{\textit{#1}}}}
\newcommand{\ErrorTok}[1]{\textcolor[rgb]{0.64,0.00,0.00}{\textbf{#1}}}
\newcommand{\ExtensionTok}[1]{#1}
\newcommand{\FloatTok}[1]{\textcolor[rgb]{0.00,0.00,0.81}{#1}}
\newcommand{\FunctionTok}[1]{\textcolor[rgb]{0.13,0.29,0.53}{\textbf{#1}}}
\newcommand{\ImportTok}[1]{#1}
\newcommand{\InformationTok}[1]{\textcolor[rgb]{0.56,0.35,0.01}{\textbf{\textit{#1}}}}
\newcommand{\KeywordTok}[1]{\textcolor[rgb]{0.13,0.29,0.53}{\textbf{#1}}}
\newcommand{\NormalTok}[1]{#1}
\newcommand{\OperatorTok}[1]{\textcolor[rgb]{0.81,0.36,0.00}{\textbf{#1}}}
\newcommand{\OtherTok}[1]{\textcolor[rgb]{0.56,0.35,0.01}{#1}}
\newcommand{\PreprocessorTok}[1]{\textcolor[rgb]{0.56,0.35,0.01}{\textit{#1}}}
\newcommand{\RegionMarkerTok}[1]{#1}
\newcommand{\SpecialCharTok}[1]{\textcolor[rgb]{0.81,0.36,0.00}{\textbf{#1}}}
\newcommand{\SpecialStringTok}[1]{\textcolor[rgb]{0.31,0.60,0.02}{#1}}
\newcommand{\StringTok}[1]{\textcolor[rgb]{0.31,0.60,0.02}{#1}}
\newcommand{\VariableTok}[1]{\textcolor[rgb]{0.00,0.00,0.00}{#1}}
\newcommand{\VerbatimStringTok}[1]{\textcolor[rgb]{0.31,0.60,0.02}{#1}}
\newcommand{\WarningTok}[1]{\textcolor[rgb]{0.56,0.35,0.01}{\textbf{\textit{#1}}}}

% tightlist command for lists without linebreak
\providecommand{\tightlist}{%
  \setlength{\itemsep}{0pt}\setlength{\parskip}{0pt}}



\usepackage{booktabs}
\usepackage{longtable}

\begin{document}



%%%%%%%%%%%%%%%%%%%%%%%%%%%%%%%%%%%%%%%%%%

\hypertarget{introducciuxf3n}{%
\section{1. Introducción}\label{introducciuxf3n}}

En la actualidad, la digitalización de los tickets de compra se ha
convertido en una práctica común en grandes cadenas de supermercados.
Estos tickets electrónicos, enviados en formato PDF al correo del
cliente, no solo reducen el uso de papel, sino que también generan datos
valiosos que pueden ser analizados para obtener información relevante
sobre los hábitos de consumo, la evolución de precios y las preferencias
de los compradores.

Este proyecto tiene como objetivo desarrollar un sistema de análisis
automatizado que permita extraer, procesar y visualizar la información
contenida en los tickets de Mercadona. Mediante técnicas de tratamiento
de datos en R, exploraremos patrones de compra, identificaremos los
productos más vendidos, analizaremos la evolución temporal de los
precios y determinaremos tendencias en función de la ubicación de la
tienda y el momento de la compra.

\hypertarget{material-y-muxe9todos}{%
\section{2. Material y Métodos}\label{material-y-muxe9todos}}

Para llevar a cabo este proyecto se han seleccionando un conjunto de
librerías específicas que respondieran a los distintos requerimientos
del análisis.

Para la manipulación de datos se emplearon los paquetes tidyverse
(incluyendo dplyr y stringr), que permitieron realizar operaciones de
filtrado, transformación y procesamiento de texto de manera eficiente.
La extracción del contenido textual desde los archivos PDF se realizó
mediante el paquete pdftools, capaz de preservar la estructura original
de los documentos.

Las visualizaciones se generaron utilizando ggplot2, seleccionado por su
versatilidad para crear gráficos de alta calidad. Para la presentación
de resultados en formatos reproducibles se implementó knitr, facilitando
la integración de código, resultados y texto explicativo.

Utilizaremos dos data frames para manejar los datos de manera más
eficiente. El primer data frame contendrá la información general del
ticket, como la dirección del supermercado, la fecha y hora de la
compra, el monto total, entre otros. En este caso, todos los productos
registrados en el ticket se almacenarán como una sola cadena de texto en
una única variable. El segundo data frame desglosará los productos en
variables separadas

Ambos data frames estarán vinculados a través de la variable fs (factura
simplificada).

\hypertarget{importaciuxf3n-de-los-datos}{%
\section{3. Importación de los
datos}\label{importaciuxf3n-de-los-datos}}

\hypertarget{carga-de-ficheros}{%
\subsection{3.1 Carga de ficheros}\label{carga-de-ficheros}}

Para evitar errores durante el procesamiento posterior, se realizó un
cambio en los nombres de los archivos PDF originales. Los archivos
fueron renombrados de forma secuencial con un formato estándar. Esta
acción se llevó a cabo una única vez, y por ello el código
correspondiente fue comentado en el script, con el fin de prevenir que
los archivos se sobrescriban accidentalmente al ejecutar el programa más
de una vez.

Se procedió a cargar automáticamente todos los archivos PDF contenidos
en la carpeta de trabajo designada. Para ello, se empleó una función que
permite listar únicamente los archivos con extensión .pdf, garantizando
así que solo se consideren los documentos relevantes para el análisis.
Esta carga automatizada facilita el procesamiento por lotes y evita la
necesidad de seleccionar manualmente cada archivo.

Para el proyecto se han importado un total de 303 archivos.

\hypertarget{carga-de-datos}{%
\subsection{3.2 Carga de datos}\label{carga-de-datos}}

Se construyó un data frame a partir de los datos extraídos, realizando
las transformaciones necesarias para asegurar que cada variable tuviera
el formato y tipo de dato adecuados.

Para la lectura de los archivos PDF se utilizaron funciones de la
librería pdftools, mientras que la limpieza y manipulación de las
cadenas de texto se llevó a cabo con funciones de la librería stringr.

Durante el procesamiento, se asignó el tipo de dato Date a la variable
de fecha y se convirtieron en valores numéricos las variables decimales
como el total de la compra, la base imponible y la cuota de IVA. Cabe
señalar que muchos de los datos no pueden incorporarse directamente al
data frame, ya que en el formato original aparecen combinados en una
misma línea. Es el caso de la fecha, la hora y el número de operación,
que debieron separarse y asignarse a variables distintas.

Por motivos de espacio y legibilidad, no se incluye la visualización del
data frame generado a partir de los tickets.

\begin{Shaded}
\begin{Highlighting}[]
\NormalTok{type }\OtherTok{\textless{}{-}} \FunctionTok{c}\NormalTok{(}\StringTok{"comercio"}\NormalTok{,}\StringTok{"empresa"}\NormalTok{, }\StringTok{"direccion"}\NormalTok{, }\StringTok{"cp"}\NormalTok{, }\StringTok{"telefono"}\NormalTok{, }\StringTok{"fecha"}\NormalTok{, }\StringTok{"hora"}\NormalTok{, }\StringTok{"op"}\NormalTok{, }\StringTok{"fs"}\NormalTok{, }\StringTok{"productos"}\NormalTok{, }\StringTok{"total"}\NormalTok{, }\StringTok{"forma\_pago"}\NormalTok{, }\StringTok{"base\_imp"}\NormalTok{, }\StringTok{"cuota"}\NormalTok{, }\StringTok{"nombre del comercio"}\NormalTok{, }\StringTok{"tipo y código de empresa"}\NormalTok{, }\StringTok{"dirección del comercio"}\NormalTok{, }\StringTok{"código postal"}\NormalTok{, }\StringTok{"teléfono del comercio"}\NormalTok{, }\StringTok{"fecha de la compra (día{-}mes{-}año)"}\NormalTok{, }\StringTok{"hora de la compra (horas y minutos)"}\NormalTok{, }\StringTok{"número del código de la operación"}\NormalTok{, }\StringTok{"código de la factura simplificada"}\NormalTok{, }\StringTok{"lista con los productos comprados"}\NormalTok{, }\StringTok{"dinero total de la compra"}\NormalTok{, }\StringTok{"forma de pago (tarjeta o en efectivo)"}\NormalTok{, }\StringTok{"base imponible (IVA)"}\NormalTok{, }\StringTok{"cuota del IVA"}\NormalTok{)}
\NormalTok{knitr}\SpecialCharTok{::}\FunctionTok{kable}\NormalTok{(rticles}\SpecialCharTok{::}\FunctionTok{string\_to\_table}\NormalTok{(type, }\DecValTok{2}\NormalTok{),}\AttributeTok{align =} \StringTok{\textquotesingle{}c\textquotesingle{}}\NormalTok{, }
             \AttributeTok{col.names =} \FunctionTok{c}\NormalTok{(}\StringTok{"Variable"}\NormalTok{, }\StringTok{"Descripción"}\NormalTok{),}
             \AttributeTok{format =} \StringTok{"latex"}\NormalTok{, }\AttributeTok{booktab =} \ConstantTok{TRUE}\NormalTok{, }
             \AttributeTok{caption =} \StringTok{"Descripción de variables"}\NormalTok{) }
\end{Highlighting}
\end{Shaded}

\begin{table}

\caption{\label{tab:tabla-variables}Descripción de variables}
\centering
\begin{tabular}[t]{cc}
\toprule
Variable & Descripción\\
\midrule
comercio & nombre del comercio\\
empresa & tipo y código de empresa\\
direccion & dirección del comercio\\
cp & código postal\\
telefono & teléfono del comercio\\
\addlinespace
fecha & fecha de la compra (día-mes-año)\\
hora & hora de la compra (horas y minutos)\\
op & número del código de la operación\\
fs & código de la factura simplificada\\
productos & lista con los productos comprados\\
\addlinespace
total & dinero total de la compra\\
forma\_pago & forma de pago (tarjeta o en efectivo)\\
base\_imp & base imponible (IVA)\\
cuota & cuota del IVA\\
\bottomrule
\end{tabular}
\end{table}

Las variables finales obtenidas se presentan en la Tabla
@ref(tab:tabla-variables).

\begin{verbatim}
##           archivo aparcamiento
## 1   ./data/M1.pdf           NA
## 2  ./data/M10.pdf           NA
## 3 ./data/M100.pdf           NA
## 4 ./data/M101.pdf           NA
## 5 ./data/M102.pdf           NA
## 6 ./data/M103.pdf           NA
\end{verbatim}

\hypertarget{analizamos-los-productos}{%
\section{4. Analizamos los productos}\label{analizamos-los-productos}}

La información relativa a los productos comprados se encontraba
inicialmente agrupada dentro de una única columna del data frame
principal. Para facilitar su análisis, se extrajo esta columna a un
nuevo data frame, separando los productos que venían concatenados en una
misma celda.

\hypertarget{productos-pescateria}{%
\subsection{4.1 Productos pescateria}\label{productos-pescateria}}

El procesamiento de los productos se realizó en varias etapas, según el
tipo de producto y la forma en que aparecían en el ticket. En primer
lugar, se identificaron los productos de pescadería, que siguen un
formato particular: aparecen siempre precedidos por una línea con la
palabra ``PESCADO'', seguida del nombre del producto en la línea
siguiente y de los detalles de compra (peso, precio por kilo y total) en
una tercera línea. A partir de esta estructura, se extrajeron los datos
relevantes y se almacenaron en un nuevo data frame específico para
pescado.

\hypertarget{productos-vendidos-por-peso}{%
\subsection{4.2 Productos vendidos por
peso}\label{productos-vendidos-por-peso}}

Después, se eliminaron las filas correspondientes a productos de
pescadería para poder trabajar exclusivamente con los productos que
también se venden por peso, como frutas y verduras. Estos artículos
generalmente constan de dos líneas: la primera contiene el nombre del
producto y la segunda incluye el peso, el precio por kilogramo y el
importe total. A partir de esta estructura se construyó un segundo data
frame con las frutas y verduras, extrayendo y transformando la
información necesaria.

\hypertarget{productos-vendidos-por-unidad}{%
\subsection{4.2 Productos vendidos por
unidad}\label{productos-vendidos-por-unidad}}

Una vez separados los productos por peso, se procedió a procesar el
resto de productos, es decir, aquellos que se venden por unidades. En
este caso, se extrajeron datos como la cantidad, el nombre del producto,
el precio unitario y el importe total. Se aplicaron técnicas de
procesamiento de texto para limpiar y estructurar la información, ya que
algunos productos con una sola unidad incluían el precio directamente
dentro del nombre del producto.

\hypertarget{productos-analizados}{%
\section{4.4 productos analizados}\label{productos-analizados}}

Finalmente, los tres grupos de productos ---pescado, frutas y verduras,
y productos por unidades--- se combinaron en un único data frame
unificado. A este conjunto se le añadió una columna adicional que
indicaba si el ticket incluía un servicio de aparcamiento, en caso de
que esa información estuviera disponible. El resultado fue un data frame
final, estructurado y homogéneo, con todos los productos organizados por
tipo, cantidad, precio, importe y número de ticket, listo para su
análisis posterior.

\begin{Shaded}
\begin{Highlighting}[]
\NormalTok{df\_producto }\OtherTok{\textless{}{-}}\NormalTok{ df }\SpecialCharTok{\%\textgreater{}\%} \FunctionTok{select}\NormalTok{(}\FunctionTok{c}\NormalTok{(num\_ticket, productos)) }\SpecialCharTok{\%\textgreater{}\%} 
  \FunctionTok{separate\_rows}\NormalTok{(productos, }\AttributeTok{sep =} \StringTok{";"}\NormalTok{)}
\end{Highlighting}
\end{Shaded}

\begin{Shaded}
\begin{Highlighting}[]
\CommentTok{\# Procesamiento de pescado por kg}

\CommentTok{\# Identificar filas con "PESCADO"}
\NormalTok{filas\_pescado }\OtherTok{\textless{}{-}} \FunctionTok{which}\NormalTok{(df\_producto}\SpecialCharTok{$}\NormalTok{productos }\SpecialCharTok{==} \StringTok{"PESCADO"}\NormalTok{)}

\CommentTok{\# Inicializar vectores para almacenar datos}
\NormalTok{num\_ticket\_vec }\OtherTok{\textless{}{-}} \FunctionTok{character}\NormalTok{()}
\NormalTok{nombre\_producto\_vec }\OtherTok{\textless{}{-}} \FunctionTok{character}\NormalTok{()}
\NormalTok{peso\_kg\_vec }\OtherTok{\textless{}{-}} \FunctionTok{numeric}\NormalTok{()}
\NormalTok{precio\_kg\_vec }\OtherTok{\textless{}{-}} \FunctionTok{numeric}\NormalTok{()}
\NormalTok{precio\_total\_vec }\OtherTok{\textless{}{-}} \FunctionTok{numeric}\NormalTok{()}

\ControlFlowTok{for}\NormalTok{ (i }\ControlFlowTok{in} \FunctionTok{seq\_along}\NormalTok{(filas\_pescado)) \{}
\NormalTok{  idx }\OtherTok{\textless{}{-}}\NormalTok{ filas\_pescado[i]}
  
  \CommentTok{\# Extraer información básica}
\NormalTok{  num\_ticket }\OtherTok{\textless{}{-}}\NormalTok{ df\_producto}\SpecialCharTok{$}\NormalTok{num\_ticket[idx]}
\NormalTok{  nombre }\OtherTok{\textless{}{-}}\NormalTok{ df\_producto}\SpecialCharTok{$}\NormalTok{productos[idx }\SpecialCharTok{+} \DecValTok{1}\NormalTok{]  }\CommentTok{\# Nombre en la siguiente fila}
  
  \CommentTok{\# Procesar la fila de detalles}
\NormalTok{  detalles }\OtherTok{\textless{}{-}}\NormalTok{ df\_producto}\SpecialCharTok{$}\NormalTok{productos[idx }\SpecialCharTok{+} \DecValTok{2}\NormalTok{]}
  
  \CommentTok{\# Limpiar y dividir la cadena de detalles}
\NormalTok{  detalles\_limpio }\OtherTok{\textless{}{-}} \FunctionTok{gsub}\NormalTok{(}\StringTok{","}\NormalTok{, }\StringTok{"."}\NormalTok{, detalles)  }\CommentTok{\# Reemplazar comas por puntos}
\NormalTok{  detalles\_split }\OtherTok{\textless{}{-}} \FunctionTok{strsplit}\NormalTok{(}\FunctionTok{trimws}\NormalTok{(detalles\_limpio), }\StringTok{"}\SpecialCharTok{\textbackslash{}\textbackslash{}}\StringTok{s+"}\NormalTok{)[[}\DecValTok{1}\NormalTok{]]}
  
  \CommentTok{\# Extraer valores (asumiendo orden: peso, unidad, precio\_kg, moneda, precio\_total)}
  \ControlFlowTok{if}\NormalTok{ (}\FunctionTok{length}\NormalTok{(detalles\_split) }\SpecialCharTok{\textgreater{}=} \DecValTok{5}\NormalTok{) \{}
\NormalTok{    peso\_kg }\OtherTok{\textless{}{-}} \FunctionTok{as.numeric}\NormalTok{(detalles\_split[}\DecValTok{1}\NormalTok{])}
\NormalTok{    precio\_kg }\OtherTok{\textless{}{-}} \FunctionTok{as.numeric}\NormalTok{(detalles\_split[}\DecValTok{3}\NormalTok{])}
\NormalTok{    importe }\OtherTok{\textless{}{-}} \FunctionTok{as.numeric}\NormalTok{(detalles\_split[}\DecValTok{5}\NormalTok{])}
    
    \CommentTok{\# Almacenar en vectores}
\NormalTok{    num\_ticket\_vec }\OtherTok{\textless{}{-}} \FunctionTok{c}\NormalTok{(num\_ticket\_vec, num\_ticket)}
\NormalTok{    nombre\_producto\_vec }\OtherTok{\textless{}{-}} \FunctionTok{c}\NormalTok{(nombre\_producto\_vec, nombre)}
\NormalTok{    peso\_kg\_vec }\OtherTok{\textless{}{-}} \FunctionTok{c}\NormalTok{(peso\_kg\_vec, peso\_kg)}
\NormalTok{    precio\_kg\_vec }\OtherTok{\textless{}{-}} \FunctionTok{c}\NormalTok{(precio\_kg\_vec, precio\_kg)}
\NormalTok{    precio\_total\_vec }\OtherTok{\textless{}{-}} \FunctionTok{c}\NormalTok{(precio\_total\_vec, importe)}
\NormalTok{  \}}
\NormalTok{\}}

\NormalTok{df\_pescado }\OtherTok{\textless{}{-}} \FunctionTok{data.frame}\NormalTok{(}
  \AttributeTok{num\_ticket =}\NormalTok{ num\_ticket\_vec,}
  \AttributeTok{nombre =}\NormalTok{ nombre\_producto\_vec,}
  \AttributeTok{peso\_kg =}\NormalTok{ peso\_kg\_vec,}
  \AttributeTok{precio\_kg =}\NormalTok{ precio\_kg\_vec,}
  \AttributeTok{importe =}\NormalTok{ precio\_total\_vec,}
  \AttributeTok{stringsAsFactors =} \ConstantTok{FALSE}
\NormalTok{)}
\end{Highlighting}
\end{Shaded}

\begin{Shaded}
\begin{Highlighting}[]
\CommentTok{\# Procesamiento fruta y la verdura}

\CommentTok{\#Identificar TODOS los bloques de pescado (3 filas cada uno)}
\NormalTok{bloques\_pescado }\OtherTok{\textless{}{-}} \FunctionTok{which}\NormalTok{(df\_producto}\SpecialCharTok{$}\NormalTok{productos }\SpecialCharTok{==} \StringTok{"PESCADO"}\NormalTok{)}

\CommentTok{\# Crear vector con TODAS las filas a eliminar (cada bloque son 3 filas)}
\NormalTok{filas\_a\_eliminar }\OtherTok{\textless{}{-}} \FunctionTok{unlist}\NormalTok{(}\FunctionTok{lapply}\NormalTok{(bloques\_pescado, }\ControlFlowTok{function}\NormalTok{(x) x}\SpecialCharTok{:}\NormalTok{(x}\SpecialCharTok{+}\DecValTok{2}\NormalTok{)))}

\CommentTok{\# Eliminar todos los bloques}
\NormalTok{df\_sin\_pescado }\OtherTok{\textless{}{-}}\NormalTok{ df\_producto[}\SpecialCharTok{{-}}\NormalTok{filas\_a\_eliminar, ]}

\NormalTok{ind\_detalles\_kg }\OtherTok{\textless{}{-}} \FunctionTok{grep}\NormalTok{(}\StringTok{"kg.*€/kg"}\NormalTok{, df\_sin\_pescado}\SpecialCharTok{$}\NormalTok{productos, }\AttributeTok{value =} \ConstantTok{FALSE}\NormalTok{)}

\NormalTok{df\_fruta\_verdura }\OtherTok{\textless{}{-}} \FunctionTok{data.frame}\NormalTok{(}
  \AttributeTok{num\_ticket =}\NormalTok{ df\_sin\_pescado}\SpecialCharTok{$}\NormalTok{num\_ticket[ind\_detalles\_kg],}
  \AttributeTok{nombre =}\NormalTok{ df\_sin\_pescado}\SpecialCharTok{$}\NormalTok{productos[ind\_detalles\_kg }\SpecialCharTok{{-}} \DecValTok{1}\NormalTok{],}
  \AttributeTok{detalles =}\NormalTok{ df\_sin\_pescado}\SpecialCharTok{$}\NormalTok{productos[ind\_detalles\_kg],}
  \AttributeTok{stringsAsFactors =} \ConstantTok{FALSE}
\NormalTok{) }\SpecialCharTok{\%\textgreater{}\%}
\FunctionTok{mutate}\NormalTok{(}
  \CommentTok{\# Limpiar el nombre (eliminar números iniciales)}
  \AttributeTok{nombre =} \FunctionTok{gsub}\NormalTok{(}\StringTok{"\^{}}\SpecialCharTok{\textbackslash{}\textbackslash{}}\StringTok{d+}\SpecialCharTok{\textbackslash{}\textbackslash{}}\StringTok{s*"}\NormalTok{, }\StringTok{""}\NormalTok{, nombre),}
  
  \CommentTok{\# Extraer peso (kg) {-} primer número en la línea}
  \AttributeTok{peso\_kg =} \FunctionTok{as.numeric}\NormalTok{(}\FunctionTok{gsub}\NormalTok{(}\StringTok{","}\NormalTok{, }\StringTok{"."}\NormalTok{, }\FunctionTok{str\_extract}\NormalTok{(detalles, }\StringTok{"\^{}[0{-}9,]+"}\NormalTok{))),}
  
  \CommentTok{\# Extraer precio por kg {-} método mejorado}
  \AttributeTok{precio\_kg =} \FunctionTok{as.numeric}\NormalTok{(}\FunctionTok{gsub}\NormalTok{(}\StringTok{","}\NormalTok{, }\StringTok{"."}\NormalTok{, }
    \FunctionTok{str\_extract}\NormalTok{(detalles, }\StringTok{"[0{-}9,]+(?=}\SpecialCharTok{\textbackslash{}\textbackslash{}}\StringTok{s*€/kg)"}\NormalTok{))),}
  
  \CommentTok{\# Extraer importe total {-} último número en la línea}
  \AttributeTok{importe =} \FunctionTok{as.numeric}\NormalTok{(}\FunctionTok{gsub}\NormalTok{(}\StringTok{","}\NormalTok{, }\StringTok{"."}\NormalTok{, }
    \FunctionTok{str\_extract}\NormalTok{(detalles, }\StringTok{"[0{-}9,]+$"}\NormalTok{)))}
\NormalTok{) }\SpecialCharTok{\%\textgreater{}\%}
\FunctionTok{select}\NormalTok{(}\SpecialCharTok{{-}}\NormalTok{detalles)}
\end{Highlighting}
\end{Shaded}

\begin{Shaded}
\begin{Highlighting}[]
\CommentTok{\# Procesamiento resto de productos sin kg}

\CommentTok{\# Identificar las filas de detalles (kg y €/kg)}
\NormalTok{ind\_detalles\_kg }\OtherTok{\textless{}{-}} \FunctionTok{grep}\NormalTok{(}\StringTok{"kg.*€/kg"}\NormalTok{, df\_sin\_pescado}\SpecialCharTok{$}\NormalTok{productos, }\AttributeTok{value =} \ConstantTok{FALSE}\NormalTok{)}

\CommentTok{\# Las filas de nombres están justo antes de los detalles}
\NormalTok{ind\_nombres\_kg }\OtherTok{\textless{}{-}}\NormalTok{ ind\_detalles\_kg }\SpecialCharTok{{-}} \DecValTok{1}

\CommentTok{\# Combinar todos los índices a eliminar}
\NormalTok{filas\_fruta\_verdura }\OtherTok{\textless{}{-}} \FunctionTok{sort}\NormalTok{(}\FunctionTok{unique}\NormalTok{(}\FunctionTok{c}\NormalTok{(ind\_nombres\_kg, ind\_detalles\_kg)))}

\CommentTok{\# Eliminar filas ya procesadas}
\NormalTok{df\_resto }\OtherTok{\textless{}{-}}\NormalTok{ df\_sin\_pescado[}\SpecialCharTok{{-}}\NormalTok{filas\_fruta\_verdura, ]}

\NormalTok{df\_productos\_unidades }\OtherTok{\textless{}{-}}\NormalTok{ df\_resto }\SpecialCharTok{\%\textgreater{}\%}
  \FunctionTok{mutate}\NormalTok{(}
    \CommentTok{\# 1. Extraer cantidad (siempre es el primer número)}
    \AttributeTok{cantidad =} \FunctionTok{as.numeric}\NormalTok{(}\FunctionTok{str\_extract}\NormalTok{(productos, }\StringTok{"\^{}}\SpecialCharTok{\textbackslash{}\textbackslash{}}\StringTok{d+"}\NormalTok{)),}
    
    \CommentTok{\# 2. Extraer posible precio en el nombre (para productos de 1 unidad)}
    \AttributeTok{precio\_en\_nombre =} \FunctionTok{ifelse}\NormalTok{(cantidad }\SpecialCharTok{==} \DecValTok{1}\NormalTok{,}
                            \FunctionTok{as.numeric}\NormalTok{(}\FunctionTok{gsub}\NormalTok{(}\StringTok{","}\NormalTok{, }\StringTok{"."}\NormalTok{, }\FunctionTok{str\_extract}\NormalTok{(productos, }\StringTok{"}\SpecialCharTok{\textbackslash{}\textbackslash{}}\StringTok{d+,}\SpecialCharTok{\textbackslash{}\textbackslash{}}\StringTok{d+$"}\NormalTok{))),}
                            \ConstantTok{NA\_real\_}\NormalTok{),}
    
    \CommentTok{\# 3. Procesamiento vectorizado de componentes}
    \AttributeTok{componentes =} \FunctionTok{strsplit}\NormalTok{(productos, }\StringTok{"}\SpecialCharTok{\textbackslash{}\textbackslash{}}\StringTok{s+"}\NormalTok{),}
    
    \CommentTok{\# 4. Extraer importe normal (para productos con múltiples unidades)}
    \AttributeTok{importe\_normal =} \FunctionTok{sapply}\NormalTok{(componentes, }\ControlFlowTok{function}\NormalTok{(x) \{}
      \ControlFlowTok{if}\NormalTok{(}\FunctionTok{length}\NormalTok{(x) }\SpecialCharTok{\textgreater{}=} \DecValTok{3}\NormalTok{) }\FunctionTok{as.numeric}\NormalTok{(}\FunctionTok{gsub}\NormalTok{(}\StringTok{","}\NormalTok{, }\StringTok{"."}\NormalTok{, x[}\FunctionTok{length}\NormalTok{(x)])) }\ControlFlowTok{else} \ConstantTok{NA\_real\_}
\NormalTok{    \}),}
    
    \CommentTok{\# 5. Determinar el importe final}
    \AttributeTok{importe =} \FunctionTok{ifelse}\NormalTok{(}\SpecialCharTok{!}\FunctionTok{is.na}\NormalTok{(precio\_en\_nombre), precio\_en\_nombre, importe\_normal),}
    
    \CommentTok{\# 6. Extraer descripción limpia (MODIFICACIÓN CLAVE)}
    \AttributeTok{nombre =} \FunctionTok{mapply}\NormalTok{(}\ControlFlowTok{function}\NormalTok{(comp, prod, cant, precio\_nombre) \{}
      \CommentTok{\# Primero eliminar la cantidad inicial (si existe)}
\NormalTok{      nombre\_limpio }\OtherTok{\textless{}{-}} \FunctionTok{gsub}\NormalTok{(}\StringTok{"\^{}}\SpecialCharTok{\textbackslash{}\textbackslash{}}\StringTok{d+}\SpecialCharTok{\textbackslash{}\textbackslash{}}\StringTok{s*"}\NormalTok{, }\StringTok{""}\NormalTok{, prod)}
      
      \ControlFlowTok{if}\NormalTok{(}\FunctionTok{length}\NormalTok{(comp) }\SpecialCharTok{\textless{}=} \DecValTok{2}\NormalTok{) }\FunctionTok{return}\NormalTok{(nombre\_limpio)  }\CommentTok{\# Caso simple}
      
      \ControlFlowTok{if}\NormalTok{(}\SpecialCharTok{!}\FunctionTok{is.na}\NormalTok{(precio\_nombre)) \{}
        \CommentTok{\# Para productos de 1 unidad: eliminar precio final}
        \FunctionTok{gsub}\NormalTok{(}\StringTok{"}\SpecialCharTok{\textbackslash{}\textbackslash{}}\StringTok{s+}\SpecialCharTok{\textbackslash{}\textbackslash{}}\StringTok{d+,}\SpecialCharTok{\textbackslash{}\textbackslash{}}\StringTok{d+$"}\NormalTok{, }\StringTok{""}\NormalTok{, nombre\_limpio)}
\NormalTok{      \} }\ControlFlowTok{else}\NormalTok{ \{}
        \CommentTok{\# Para múltiples unidades: eliminar elementos numéricos finales}
        \FunctionTok{paste}\NormalTok{(comp[}\DecValTok{2}\SpecialCharTok{:}\NormalTok{(}\FunctionTok{length}\NormalTok{(comp)}\SpecialCharTok{{-}}\DecValTok{2}\NormalTok{)], }\AttributeTok{collapse=}\StringTok{" "}\NormalTok{)}
\NormalTok{      \}}
\NormalTok{    \}, componentes, productos, cantidad, precio\_en\_nombre, }\AttributeTok{SIMPLIFY =} \ConstantTok{TRUE}\NormalTok{) }\SpecialCharTok{\%\textgreater{}\%}
      \FunctionTok{str\_trim}\NormalTok{()  }\CommentTok{\# Eliminar espacios sobrantes}
\NormalTok{  ) }\SpecialCharTok{\%\textgreater{}\%}
  \FunctionTok{mutate}\NormalTok{(}
    \CommentTok{\# 7. Calcular precio unitario}
    \AttributeTok{precio\_unitario =}\NormalTok{ importe }\SpecialCharTok{/}\NormalTok{ cantidad}
\NormalTok{  ) }\SpecialCharTok{\%\textgreater{}\%}
  \FunctionTok{select}\NormalTok{(num\_ticket, nombre, cantidad, precio\_unitario, importe)}
\end{Highlighting}
\end{Shaded}

\begin{verbatim}
## Warning: There were 33 warnings in `mutate()`.
## The first warning was:
## i In argument: `importe_normal = sapply(...)`.
## Caused by warning in `FUN()`:
## ! NAs introducidos por coerción
## i Run `dplyr::last_dplyr_warnings()` to see the 32 remaining warnings.
\end{verbatim}

\begin{Shaded}
\begin{Highlighting}[]
\CommentTok{\# dataframe final con todos los productos analizados:}

\CommentTok{\# Añadir columna \textquotesingle{}tipo\textquotesingle{} a cada dataframe}
\NormalTok{df\_pescado }\OtherTok{\textless{}{-}}\NormalTok{ df\_pescado }\SpecialCharTok{\%\textgreater{}\%} \FunctionTok{mutate}\NormalTok{(}\AttributeTok{tipo =} \StringTok{"pescado"}\NormalTok{)}
\NormalTok{df\_fruta\_verdura }\OtherTok{\textless{}{-}}\NormalTok{ df\_fruta\_verdura }\SpecialCharTok{\%\textgreater{}\%} \FunctionTok{mutate}\NormalTok{(}\AttributeTok{tipo =} \StringTok{"fruta\_verdura"}\NormalTok{)}
\NormalTok{df\_productos\_unidades }\OtherTok{\textless{}{-}}\NormalTok{ df\_productos\_unidades }\SpecialCharTok{\%\textgreater{}\%} \FunctionTok{mutate}\NormalTok{(}\AttributeTok{tipo =} \StringTok{"unidades"}\NormalTok{)}

\CommentTok{\# Unificar columnas para combinar}
\NormalTok{df\_final }\OtherTok{\textless{}{-}} \FunctionTok{bind\_rows}\NormalTok{(}
\NormalTok{  df\_pescado }\SpecialCharTok{\%\textgreater{}\%} \FunctionTok{select}\NormalTok{(num\_ticket, nombre, }\AttributeTok{cantidad =}\NormalTok{ peso\_kg, }\AttributeTok{precio =}\NormalTok{ precio\_kg, importe, tipo),}
\NormalTok{  df\_fruta\_verdura }\SpecialCharTok{\%\textgreater{}\%} \FunctionTok{select}\NormalTok{(num\_ticket, nombre, }\AttributeTok{cantidad =}\NormalTok{ peso\_kg, }\AttributeTok{precio =}\NormalTok{ precio\_kg, importe, tipo),}
\NormalTok{  df\_productos\_unidades }\SpecialCharTok{\%\textgreater{}\%} \FunctionTok{select}\NormalTok{(num\_ticket, }\AttributeTok{nombre =}\NormalTok{ nombre, cantidad, }\AttributeTok{precio =}\NormalTok{ precio\_unitario, importe, tipo)}
\NormalTok{)}

\CommentTok{\# Crear columna tiene\_aparcamiento }
\ControlFlowTok{if}\NormalTok{ (}\StringTok{"aparcamiento"} \SpecialCharTok{\%in\%} \FunctionTok{colnames}\NormalTok{(df\_aparcamiento)) \{}
\NormalTok{  df\_aparcamiento }\OtherTok{\textless{}{-}}\NormalTok{ df\_aparcamiento }\SpecialCharTok{\%\textgreater{}\%}
    \FunctionTok{mutate}\NormalTok{(}\AttributeTok{nombre\_archivo =} \FunctionTok{basename}\NormalTok{(archivo)) }\SpecialCharTok{\%\textgreater{}\%}
    \FunctionTok{mutate}\NormalTok{(}\AttributeTok{num\_ticket =} \FunctionTok{str\_extract}\NormalTok{(nombre\_archivo, }\StringTok{"}\SpecialCharTok{\textbackslash{}\textbackslash{}}\StringTok{d+"}\NormalTok{)) }\SpecialCharTok{\%\textgreater{}\%}
    \FunctionTok{mutate}\NormalTok{(}\AttributeTok{tiene\_aparcamiento =} \SpecialCharTok{!}\FunctionTok{is.na}\NormalTok{(aparcamiento)) }\SpecialCharTok{\%\textgreater{}\%}
    \FunctionTok{select}\NormalTok{(num\_ticket, tiene\_aparcamiento)}

  \CommentTok{\# Convertir num\_ticket a carácter en ambos dataframes}
\NormalTok{  df\_final }\OtherTok{\textless{}{-}}\NormalTok{ df\_final }\SpecialCharTok{\%\textgreater{}\%} \FunctionTok{mutate}\NormalTok{(}\AttributeTok{num\_ticket =} \FunctionTok{as.character}\NormalTok{(num\_ticket))}
\NormalTok{  df\_aparcamiento }\OtherTok{\textless{}{-}}\NormalTok{ df\_aparcamiento }\SpecialCharTok{\%\textgreater{}\%} \FunctionTok{mutate}\NormalTok{(}\AttributeTok{num\_ticket =} \FunctionTok{as.character}\NormalTok{(num\_ticket))}

  \CommentTok{\# Combinar los datos}
\NormalTok{  df\_final }\OtherTok{\textless{}{-}} \FunctionTok{left\_join}\NormalTok{(df\_final, df\_aparcamiento, }\AttributeTok{by =} \StringTok{"num\_ticket"}\NormalTok{)}
\NormalTok{\} }\ControlFlowTok{else}\NormalTok{ \{}
\NormalTok{  df\_final}\SpecialCharTok{$}\NormalTok{tiene\_aparcamiento }\OtherTok{\textless{}{-}} \ConstantTok{NA}  \CommentTok{\# columna vacía si no existe \textquotesingle{}aparcamiento\textquotesingle{}}
\NormalTok{\}}

\CommentTok{\# Resultado final ordenado}
\NormalTok{df\_final }\OtherTok{\textless{}{-}}\NormalTok{ df\_final }\SpecialCharTok{\%\textgreater{}\%} \FunctionTok{arrange}\NormalTok{(num\_ticket)}

\CommentTok{\# Verificación final}
\FunctionTok{cat}\NormalTok{(}\StringTok{"}\SpecialCharTok{\textbackslash{}n}\StringTok{Resumen del DataFrame final:}\SpecialCharTok{\textbackslash{}n}\StringTok{"}\NormalTok{)}
\end{Highlighting}
\end{Shaded}

\begin{verbatim}
## 
## Resumen del DataFrame final:
\end{verbatim}

\begin{Shaded}
\begin{Highlighting}[]
\FunctionTok{cat}\NormalTok{(}\StringTok{"{-} Pescado:"}\NormalTok{, }\FunctionTok{sum}\NormalTok{(df\_final}\SpecialCharTok{$}\NormalTok{tipo }\SpecialCharTok{==} \StringTok{"pescado"}\NormalTok{), }\StringTok{"registros}\SpecialCharTok{\textbackslash{}n}\StringTok{"}\NormalTok{)}
\end{Highlighting}
\end{Shaded}

\begin{verbatim}
## - Pescado: 28 registros
\end{verbatim}

\begin{Shaded}
\begin{Highlighting}[]
\FunctionTok{cat}\NormalTok{(}\StringTok{"{-} Fruta/Verdura:"}\NormalTok{, }\FunctionTok{sum}\NormalTok{(df\_final}\SpecialCharTok{$}\NormalTok{tipo }\SpecialCharTok{==} \StringTok{"fruta\_verdura"}\NormalTok{), }\StringTok{"registros}\SpecialCharTok{\textbackslash{}n}\StringTok{"}\NormalTok{)}
\end{Highlighting}
\end{Shaded}

\begin{verbatim}
## - Fruta/Verdura: 395 registros
\end{verbatim}

\begin{Shaded}
\begin{Highlighting}[]
\FunctionTok{cat}\NormalTok{(}\StringTok{"{-} Unidades:"}\NormalTok{, }\FunctionTok{sum}\NormalTok{(df\_final}\SpecialCharTok{$}\NormalTok{tipo }\SpecialCharTok{==} \StringTok{"unidades"}\NormalTok{), }\StringTok{"registros}\SpecialCharTok{\textbackslash{}n}\StringTok{"}\NormalTok{)}
\end{Highlighting}
\end{Shaded}

\begin{verbatim}
## - Unidades: 4702 registros
\end{verbatim}

\begin{Shaded}
\begin{Highlighting}[]
\FunctionTok{cat}\NormalTok{(}\StringTok{"{-} Total:"}\NormalTok{, }\FunctionTok{nrow}\NormalTok{(df\_final), }\StringTok{"registros}\SpecialCharTok{\textbackslash{}n}\StringTok{"}\NormalTok{)}
\end{Highlighting}
\end{Shaded}

\begin{verbatim}
## - Total: 5125 registros
\end{verbatim}

\begin{Shaded}
\begin{Highlighting}[]
\FunctionTok{print}\NormalTok{(}\FunctionTok{head}\NormalTok{(df\_final, }\DecValTok{10}\NormalTok{))}
\end{Highlighting}
\end{Shaded}

\begin{verbatim}
##    num_ticket              nombre cantidad precio importe          tipo
## 1      007267    PERA CONFERENCIA    0.854   2.55    2.18 fruta_verdura
## 2      007267    MANZ. ROJA DULCE    1.474   2.90    4.27 fruta_verdura
## 3      007267      PIZZA BOLOÑESA    1.000   2.60    2.60      unidades
## 4      007267    MIX FRUTOS ROJOS    1.000   1.66    1.66      unidades
## 5      007267   PISTO DE VERDURAS    1.000   2.30    2.30      unidades
## 6      007267    FILETE DE TRUCHA    1.000   3.92    3.92      unidades
## 7      007267        PAN SEMILLAS    1.000   1.60    1.60      unidades
## 8      007267         SNACK PIPAS    1.000   1.40    1.40      unidades
## 9      007267 +PROT NATILLA VAINI    1.000   1.75    1.75      unidades
## 10     007267     SNACK CHOCOLATE    1.000   1.50    1.50      unidades
##    tiene_aparcamiento
## 1                  NA
## 2                  NA
## 3                  NA
## 4                  NA
## 5                  NA
## 6                  NA
## 7                  NA
## 8                  NA
## 9                  NA
## 10                 NA
\end{verbatim}

\#Preguntas

¿Cuáles son los productos menos vendidos por unidades? ¿Y por kilos?

¿Qué productos han generado mayor ingreso total (precio x cantidad)?

¿Cuáles productos han aumentado o disminuido su venta a lo largo del
tiempo?

¿Qué productos se compran habitualmente juntos?

¿Existen diferencias de precios para el mismo producto en diferentes
tiendas o ubicaciones?

¿Qué días de la semana hay más ventas? ¿Y a qué horas?

¿En qué meses se venden más frutas/verduras, o pescados?

¿Desde qué ciudades se emiten más tickets?

¿Existen diferencias de consumo por ciudad?

¿Cuánto se gasta semanal o mensualmente en un supermercado?

¿Influye la disponibilidad de aparcamiento en el importe total de la
compra?

%%%%%%%%%%%%%%%%%%%%%%%%%%%%%%%%%%%%%%%%%%

\vspace{6pt}

%%%%%%%%%%%%%%%%%%%%%%%%%%%%%%%%%%%%%%%%%%
%% optional

% Only for the journal Methods and Protocols:
% If you wish to submit a video article, please do so with any other supplementary material.
% \supplementary{The following supporting information can be downloaded at: \linksupplementary{s1}, Figure S1: title; Table S1: title; Video S1: title. A supporting video article is available at doi: link.}

%%%%%%%%%%%%%%%%%%%%%%%%%%%%%%%%%%%%%%%%%%







%%%%%%%%%%%%%%%%%%%%%%%%%%%%%%%%%%%%%%%%%%
%% Optional

%% Only for journal Encyclopedia


%%%%%%%%%%%%%%%%%%%%%%%%%%%%%%%%%%%%%%%%%%
%% Optional
%%%%%%%%%%%%%%%%%%%%%%%%%%%%%%%%%%%%%%%%%%
\begin{adjustwidth}{-\extralength}{0cm}

%\printendnotes[custom] % Un-comment to print a list of endnotes



% If authors have biography, please use the format below
%\section*{Short Biography of Authors}
%\bio
%{\raisebox{-0.35cm}{\includegraphics[width=3.5cm,height=5.3cm,clip,keepaspectratio]{Definitions/author1.pdf}}}
%{\textbf{Firstname Lastname} Biography of first author}
%
%\bio
%{\raisebox{-0.35cm}{\includegraphics[width=3.5cm,height=5.3cm,clip,keepaspectratio]{Definitions/author2.jpg}}}
%{\textbf{Firstname Lastname} Biography of second author}

%%%%%%%%%%%%%%%%%%%%%%%%%%%%%%%%%%%%%%%%%%
%% for journal Sci
%\reviewreports{\\
%Reviewer 1 comments and authors’ response\\
%Reviewer 2 comments and authors’ response\\
%Reviewer 3 comments and authors’ response
%}
%%%%%%%%%%%%%%%%%%%%%%%%%%%%%%%%%%%%%%%%%%
\PublishersNote{}
\end{adjustwidth}


\end{document}
